\documentclass[a4paper,11pt,table]{article}
%Import Packages %\[
\usepackage{fullpage}                                                   %This sets the margines to standard.
\usepackage{cancel}                                                     %This allows commands to put lines through terms.
\usepackage{mathtools}                                                  %This adds a series of packages designed to enhance the appearance of documents containing a lot of mathematics.
\usepackage{amsmath,amssymb,amsfonts,amsthm}                            %This adds many math commands.
\usepackage{setspace}                                                   %This allows for double or single spacing commands.
\usepackage{tikz}                                                       %This adds drawing features.
\usepackage{pgfplots}                                                   %This adds function plotting.
\usepackage{chngcntr}                                                   %This allows for convenient numbering.
\usepackage{fancyhdr}                                                   %This allows for a nice header.
\usepackage{caption}                                                    %This allows for custom captions.
\usepackage{listings}
\usepackage{graphicx}\graphicspath{{"Final Project"/}}                     %This allows the import and display of figures. %\]
%Renew Commands %\[
\renewcommand{\t}{\text}                                                %This is a shortcut for the text environment.
\renewcommand{\c}{\cdot}                                                %This is a shortcut for the dot operator.
\renewcommand{\sp}[1]{\left[#1\right]}                                  %This is a shortcut for hard brackets.
\renewcommand{\P}{\mathbb{P}}                                           %This is a shortcut for blackboard P.
\renewcommand{\l}{\ell}                                                 %This is a shortcut for script l.
\renewcommand{\rm}[1]{\begin{bmatrix*}[r] #1\end{bmatrix*}}             %This is a shortcut for a matrix.
\renewcommand{\i}{\overline{i}}                                         %This is a shortcut for i conjugate.
\renewcommand{\j}{\overline{j}}                                         %This is a shortcut for j conjugate.
\renewcommand{\k}{\overline{k}}                                         %This is a shortcut for k conjugate.
\renewcommand{\l}{\overline{l}}                                         %This is a shortcut for k conjugate.
\renewcommand{\r}{\scriptr}                                             %This is a shortcut for script r.%\]
%New Commands %\[
\newcommand*\rfrac[2]{{}^{#1}\!/_{#2}}                                  %This creates a one line fraction.
\newcommand{\ptl}[2]{\frac{\partial #1}{\partial #2}}                   %This is a shortcut for partials.
\newcommand{\sptl}[2]{\frac{\partial^2 #1}{\partial #2^2}}              %This is a shortcut for second partials.
\newcommand{\p}[1]{\left(#1\right)}                                     %This is a shortcut for parenthesis.
\newcommand{\hb}[1]{\left[#1\right]}                                    %This is a shortcut for hard brackets.
\newcommand{\set}[1]{\left\{#1\right\}}                                 %This is a shortcut for set notation.
\newcommand{\bra}[1]{\langle #1 |}                                      %This is a shortcut for bras.
\newcommand{\ket}[1]{| #1 \rangle}                                      %This is a shortcut for Kets.
\newcommand{\R}{\mathbb{R}}                                             %This is a shortcut for blackboard R.
\newcommand{\Z}{\mathbb{Z}}                                             %This is a shortcut for blackboard Z.
\newcommand{\N}{\mathbb{N}}                                             %This is a shortcut for blackboard N.
\newcommand{\C}{\mathbb{C}}                                             %This is a shortcut for blackboard C.
\newcommand{\Q}{\mathbb{Q}}                                             %This is a shortcut for blackboard C.
\newcommand{\comm}[2]{\left[#1, #2\right]}                               %This is a shortcut for the commutation operator.
\newcommand{\F}{\mathbb{F}}                                             %This is a shortcut for blackboard F.
\newcommand{\fa}{\forall}                                               %This is a shortcut for the upsidedown A symbol.
\newcommand{\m}[1]{\begin{bmatrix*}[r] #1\end{bmatrix*}}                %This is a shortcut for a matrix.
\newcommand{\cm}[1]{\begin{bmatrix*}[c] #1\end{bmatrix*}}               %This is a shortcut for a matrix.
\newcommand{\vm}[1]{\begin{vmatrix}#1\end{vmatrix}}                     %This is a shortcut for a matrix determinant.
\newcommand{\beq}{\begin{displaymath}}                                  %This is a shortcut for opening equation editor.
\newcommand{\eeq}{\end{displaymath}}                                    %This is a shortcut for closing equation editor.
\newcommand{\xhat}{\hat{x}}                                             %This is a shortcut for the x unit vector.
\newcommand{\yhat}{\hat{y}}                                             %This is a shortcut for the y unit vector.
\newcommand{\zhat}{\hat{z}}                                             %This is a shortcut for the z unit vector.
\newcommand{\Hhat}{\hat{H}}                                             %This is a shortcut for the Hamiltonian operator.
\newcommand{\del}{\v\nabla}                                             %This is a shortcut for the del operator.
\newcommand{\norm}[1]{\left|#1\right|}                                  %This is a shortcut for the norm operation.
\newcommand{\normsq}[1]{\left|#1\right|^2}                              %This is a shortcut for the norm operation.
\DeclarePairedDelimiter\ceil{\lceil}{\rceil}                            %This is a shortcut for the ceiling shorthand.
\DeclarePairedDelimiter\floor{\lfloor}{\rfloor}                         %This is a shortcut for the floor shorthand.
\newcommand{\bvec}[1]{\mathbf{#1}}                                      %This is a shortcut for a bold character. %\]
%New Definitions %\[
%\[                                                                     %This is for folding in Notepad++
\let\oldsqrt\sqrt                                                       %This redefines the sqrt function as oldsqrt.
\def\sqrt{\mathpalette\DHLhksqrt}                                       %This line and the ones following redefine the sqrt function to be prettier.
\def\DHLhksqrt#1#2{\setbox0=\hbox{$#1\oldsqrt{#2\,}$}%
    \dimen0=\ht0\advance\dimen0-0.2\ht0\setbox2%
    =\hbox{\vrule height\ht0 depth -\dimen0}{\box0\lower0.4pt\box2}}
\def\dbar{{\mathchar'26\mkern-12mu d}}                                  %This defines the inexact differential symbol.
%\]%\]
\usepackage{hyperref}       % Hyperlink
\hypersetup{
    colorlinks=true,
    linkcolor=blue,
    filecolor=magenta,      
    urlcolor=cyan,
}

%Header %\[
\pagestyle{fancy}
\fancyhf{}
\rhead{}
\chead{From Collectivism to Individualism [and so to failure]: A Tragic Soccer Tale by Kim and Rath}
\lhead{}
\renewcommand{\headrulewidth}{0.5pt}
\setlength{\headheight}{12pt}
%\]

\renewcommand{\section}[1]{\stepcounter{section}\noindent\textbf{\arabic{section}.~#1}\\}
\counterwithin{equation}{section}
\counterwithin{figure}{section}

\begin{document}
\section{Introduction}

Predicting the outcome of sports matches is one of the favorite topics for statisticians and gamblers alike. It is an interesting problem where the question of the respective importance of randomness ('luck') and determinism ('skill') is hotly debated. Of course, much importance also lies in the fact that successful predictions can lead to lots of money. For soccer, previous prediction methods mostly used team-level features while neglecting individual skills. We hypothesize that individual skills are crucial to soccer and contain much information for predicting match outcomes which those other methods are not utilizing. Therefore, in this project, we sought to develop a machine learning strategy that predicts the outcomes of soccer matches based mainly on the individual attributes of the players on each team. \\
~\\
\section{Data Collection and Processing}

~\\
\section{Data Analysis Approach}

The first method that we used to analyze the dataset obtained by above procedure was Weka. We used various algorithms that were identified to be promising during our preliminary investigations. Specifically, those included three tree algorithms, three Bayesian algorithms, and two logistic regression algorithms.

Complementing the Weka, we decided to create our own machine learning implementation. Specifically, we wrote a \href{https://github.com/TheLordBlarg/Soccer_Success/blob/master/codes/machine_learning/msc_nn_hpc.py}{Python script} implementing a neural network with one hidden layer and three softmax output units each corresponding to the probability of the input data belonging to each class. The algorithm also uses 10-fold cross-validation with random partitioning of the dataset into training/validation sets, intentionally mirroring the Weka's approach. One thing of note is that we have normalized the data such that each attribute lies in the $[0,1]$ range (i.e. divided by 100). This seemed to improve the result significantly (increasing the classification accuray by 5-10\%), presumably due to the reduction of overflow errors. For Weka performance, it had no significant impact.

Finally, we sought to investigate the relative predictive power of each feature. This is not only interesting in its own right (we might be able to determine which skills are most relevant for soccer), it can also be a helpful aid for guiding machine learning implementation, in particular serving as a sanity-check for the approach. In order to measure this quantitatively, we wrote a \href{https://github.com/TheLordBlarg/Soccer_Success/blob/master/codes/feature_analysis/info_entropy.py}{Python script} that calculates, for each attribute, the maximum information gain that can be achieved by making a split along that attribute (cf. decision trees). \\
~\\
\section{Results}

Position-averaged, categories.

All Premier League seasons sans 2015-2016 (set aside as the test set).

\begin{center}
    \begin{tabular}{|c|c|c|c|}
        \hline
        Algorithm               &   Training accuracy  &   10-fold cross-validation accuracy  &   Test accuracy \\
        \hline
        ZeroR                   &   0.4644          &   0.4644      &   0.3919 \\
        \hline
        J48                     &   0.9439          &   0.3912      &   0.3993 \\
        \hline
        REPTree                 &   0.5741          &   0.4838      &   \cellcolor{red!25}0.4505 \\
        \hline
        RandomForest            &   \cellcolor{red!25}1.0000          &   0.4829      &   0.4396 \\
        \hline
        NaiveBayes              &   0.4049          &   0.3593      &   0.3773 \\
        \hline
        BayesNet                &   0.4967          &   0.4829      &   0.4432 \\
        \hline
        NaiveBayesMultinomial   &   0.4316          &   0.4335      &   0.4029 \\
        \hline
        Logistic                &   0.5318          &   \cellcolor{red!25}0.4948      &   0.4286 \\
        \hline
        SimpleLogistic          &   0.5119          &   0.4924      &   0.4139 \\
        \hline
    \end{tabular}
    \captionof{table}{\label{table1}}
\end{center}

Three differeng leagues.

\begin{center}
    \begin{tabular}{|c|c|c|c|}
        \hline
        Algorithm               &   Premier League  &   Bundesliga  &   La Liga \\
        \hline
        ZeroR                   &   0.4801          &   0.4896      &   0.4878 \\
        \hline
        J48                     &   0.4709          &   0.3889      &   0.4177 \\
        \hline
        REPTree                 &   0.4618          &   0.4514      &   0.5213 \\
        \hline
        RandomForest            &   \cellcolor{red!25}0.5505          &   0.4896      &   0.4787 \\
        \hline
        NaiveBayes              &   0.3700          &   0.3299      &   0.2927 \\
        \hline
        BayesNet                &   0.4954          &   0.4896      &   0.4360 \\
        \hline
        NaiveBayesMultinomial   &   0.4037          &   0.3125      &   0.4024 \\
        \hline
        Logistic                &   0.4771          &   0.4479      &   0.4390 \\
        \hline
        SimpleLogistic          &   0.5260          &   \cellcolor{red!25}0.5000      &   \cellcolor{red!25}0.5183 \\
        \hline
    \end{tabular}
    \captionof{table}{10-fold cross-validation accuracy for 2016-2017 season for three different leagues. The highest accuracy for each league has been shaded in light red. \label{tableDL}}
\end{center}

Neural network

We have trained models using different hidden layer sizes ranging from $10-100$. One example of a learning curve (for hidden layer size = 30) is shown in Figure \ref{figLC}. It can be seen clearly that the model is incapable of fitting the dataset. In fact, essentially the same results were obtained for all values of hidden layer size used (see Figure \ref{figCVH}). While this might just suggest that the neural network architecture considered is inappropriate for this problem (for instance, more than one hidden layers may be necessary), considering that none of the other algorithms used above in Weka yielded much success, it seems more likely that the problem lies with the dataset itself.

\begin{center}
    \includegraphics[scale=0.5]{learning_curve}
    \captionof{figure}{\label{figLC}}
    \includegraphics[scale=0.5]{CV_H}
    \captionof{figure}{\label{figCVH}}
\end{center}

Feature analysis

Table \ref{tablefeature} lists some of the features with the corresponding information gain, with more informative features towards the top. Full data is available \href{https://github.com/TheLordBlarg/Soccer_Success/blob/master/results/entropy_train.csv}{in the repository}.
\begin{center}
    \begin{tabular}{|c|c|c|}
        \hline
        No. &   Attribute                           &   Information gain \\
        \hline
        1   &   Home Team; Goalkeeper; Goalkeeping  &   0.03567655 \\
        \hline
        2   &   Away Team; Goalkeeper; Goalkeeping  &   0.024776643 \\
        \hline
        3   &   Home Team; Defender; Defending      &   0.013332217 \\
        \hline
        4   &   Home Team; Goalkeeper; Movement     &   0.013321315 \\
        \hline
        5   &   Home Team; Defender; Movement       &   0.011140302 \\
        \hline
        \vdots  &   \vdots  &   \vdots \\
        \hline
        37  &   Home Team; Goalkeeper; Attacking    &   0.003304952 \\
        \hline
        \vdots  &   \vdots  &   \vdots \\
        \hline
        43  &   Away Team; Goalkeeper; Attacking    &   0.002832013 \\
        \hline
        \vdots  &   \vdots  &   \vdots \\
        \hline
        51  &   Home Team; Forward; Goalkeeping     &   0.002177446 \\
        \hline
        52  &   Away Team; Forward; Goalkeeping     &   0.002115403 \\
        \hline
        53  &   Away Team; Defender; Defending      &   0.00206549 \\
        \hline
    \end{tabular}
    \captionof{table}{Prior entropy = 1.5305276569\label{tablefeature}}
\end{center}
~\\
\section{Discussion}

In retrospect, we may have been overly optimistic and ambitious in pursuing this project. Of course, soccer is a fundamentally team-based sport and neglecting crucial factors such as formations and previous team records may have doomed the prospect of success from the beginning.

~\\
\section{Conclusion}

\section{Acknowledgements}
Jeremy Rath worked mainly on data collection and processing. Hyun Jin Kim worked mainly on data analysis. Both authors worked on setting up the webpage and preparing this report.

~\\
\end{document}
